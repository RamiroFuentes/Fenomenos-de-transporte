
    




    
\documentclass[11pt]{article}

    
    \usepackage[breakable]{tcolorbox}
    \tcbset{nobeforeafter} % prevents tcolorboxes being placing in paragraphs
    \usepackage{float}
    \floatplacement{figure}{H} % forces figures to be placed at the correct location
    
    \usepackage[T1]{fontenc}
    % Nicer default font (+ math font) than Computer Modern for most use cases
    \usepackage{mathpazo}

    % Basic figure setup, for now with no caption control since it's done
    % automatically by Pandoc (which extracts ![](path) syntax from Markdown).
    \usepackage{graphicx}
    % We will generate all images so they have a width \maxwidth. This means
    % that they will get their normal width if they fit onto the page, but
    % are scaled down if they would overflow the margins.
    \makeatletter
    \def\maxwidth{\ifdim\Gin@nat@width>\linewidth\linewidth
    \else\Gin@nat@width\fi}
    \makeatother
    \let\Oldincludegraphics\includegraphics
    % Set max figure width to be 80% of text width, for now hardcoded.
    \renewcommand{\includegraphics}[1]{\Oldincludegraphics[width=.8\maxwidth]{#1}}
    % Ensure that by default, figures have no caption (until we provide a
    % proper Figure object with a Caption API and a way to capture that
    % in the conversion process - todo).
    \usepackage{caption}
    \DeclareCaptionLabelFormat{nolabel}{}
    \captionsetup{labelformat=nolabel}

    \usepackage{adjustbox} % Used to constrain images to a maximum size 
    \usepackage{xcolor} % Allow colors to be defined
    \usepackage{enumerate} % Needed for markdown enumerations to work
    \usepackage{geometry} % Used to adjust the document margins
    \usepackage{amsmath} % Equations
    \usepackage{amssymb} % Equations
    \usepackage{textcomp} % defines textquotesingle
    % Hack from http://tex.stackexchange.com/a/47451/13684:
    \AtBeginDocument{%
        \def\PYZsq{\textquotesingle}% Upright quotes in Pygmentized code
    }
    \usepackage{upquote} % Upright quotes for verbatim code
    \usepackage{eurosym} % defines \euro
    \usepackage[mathletters]{ucs} % Extended unicode (utf-8) support
    \usepackage[utf8x]{inputenc} % Allow utf-8 characters in the tex document
    \usepackage{fancyvrb} % verbatim replacement that allows latex
    \usepackage{grffile} % extends the file name processing of package graphics 
                         % to support a larger range 
    % The hyperref package gives us a pdf with properly built
    % internal navigation ('pdf bookmarks' for the table of contents,
    % internal cross-reference links, web links for URLs, etc.)
    \usepackage{hyperref}
    \usepackage{longtable} % longtable support required by pandoc >1.10
    \usepackage{booktabs}  % table support for pandoc > 1.12.2
    \usepackage[inline]{enumitem} % IRkernel/repr support (it uses the enumerate* environment)
    \usepackage[normalem]{ulem} % ulem is needed to support strikethroughs (\sout)
                                % normalem makes italics be italics, not underlines
    \usepackage{mathrsfs}
    \setlength{\parindent}{0pt}

    
    % Colors for the hyperref package
    \definecolor{urlcolor}{rgb}{0,.145,.698}
    \definecolor{linkcolor}{rgb}{.71,0.21,0.01}
    \definecolor{citecolor}{rgb}{.12,.54,.11}

    % ANSI colors
    \definecolor{ansi-black}{HTML}{3E424D}
    \definecolor{ansi-black-intense}{HTML}{282C36}
    \definecolor{ansi-red}{HTML}{E75C58}
    \definecolor{ansi-red-intense}{HTML}{B22B31}
    \definecolor{ansi-green}{HTML}{00A250}
    \definecolor{ansi-green-intense}{HTML}{007427}
    \definecolor{ansi-yellow}{HTML}{DDB62B}
    \definecolor{ansi-yellow-intense}{HTML}{B27D12}
    \definecolor{ansi-blue}{HTML}{208FFB}
    \definecolor{ansi-blue-intense}{HTML}{0065CA}
    \definecolor{ansi-magenta}{HTML}{D160C4}
    \definecolor{ansi-magenta-intense}{HTML}{A03196}
    \definecolor{ansi-cyan}{HTML}{60C6C8}
    \definecolor{ansi-cyan-intense}{HTML}{258F8F}
    \definecolor{ansi-white}{HTML}{C5C1B4}
    \definecolor{ansi-white-intense}{HTML}{A1A6B2}
    \definecolor{ansi-default-inverse-fg}{HTML}{FFFFFF}
    \definecolor{ansi-default-inverse-bg}{HTML}{000000}

    % commands and environments needed by pandoc snippets
    % extracted from the output of `pandoc -s`
    \providecommand{\tightlist}{%
      \setlength{\itemsep}{0pt}\setlength{\parskip}{0pt}}
    \DefineVerbatimEnvironment{Highlighting}{Verbatim}{commandchars=\\\{\}}
    % Add ',fontsize=\small' for more characters per line
    \newenvironment{Shaded}{}{}
    \newcommand{\KeywordTok}[1]{\textcolor[rgb]{0.00,0.44,0.13}{\textbf{{#1}}}}
    \newcommand{\DataTypeTok}[1]{\textcolor[rgb]{0.56,0.13,0.00}{{#1}}}
    \newcommand{\DecValTok}[1]{\textcolor[rgb]{0.25,0.63,0.44}{{#1}}}
    \newcommand{\BaseNTok}[1]{\textcolor[rgb]{0.25,0.63,0.44}{{#1}}}
    \newcommand{\FloatTok}[1]{\textcolor[rgb]{0.25,0.63,0.44}{{#1}}}
    \newcommand{\CharTok}[1]{\textcolor[rgb]{0.25,0.44,0.63}{{#1}}}
    \newcommand{\StringTok}[1]{\textcolor[rgb]{0.25,0.44,0.63}{{#1}}}
    \newcommand{\CommentTok}[1]{\textcolor[rgb]{0.38,0.63,0.69}{\textit{{#1}}}}
    \newcommand{\OtherTok}[1]{\textcolor[rgb]{0.00,0.44,0.13}{{#1}}}
    \newcommand{\AlertTok}[1]{\textcolor[rgb]{1.00,0.00,0.00}{\textbf{{#1}}}}
    \newcommand{\FunctionTok}[1]{\textcolor[rgb]{0.02,0.16,0.49}{{#1}}}
    \newcommand{\RegionMarkerTok}[1]{{#1}}
    \newcommand{\ErrorTok}[1]{\textcolor[rgb]{1.00,0.00,0.00}{\textbf{{#1}}}}
    \newcommand{\NormalTok}[1]{{#1}}
    
    % Additional commands for more recent versions of Pandoc
    \newcommand{\ConstantTok}[1]{\textcolor[rgb]{0.53,0.00,0.00}{{#1}}}
    \newcommand{\SpecialCharTok}[1]{\textcolor[rgb]{0.25,0.44,0.63}{{#1}}}
    \newcommand{\VerbatimStringTok}[1]{\textcolor[rgb]{0.25,0.44,0.63}{{#1}}}
    \newcommand{\SpecialStringTok}[1]{\textcolor[rgb]{0.73,0.40,0.53}{{#1}}}
    \newcommand{\ImportTok}[1]{{#1}}
    \newcommand{\DocumentationTok}[1]{\textcolor[rgb]{0.73,0.13,0.13}{\textit{{#1}}}}
    \newcommand{\AnnotationTok}[1]{\textcolor[rgb]{0.38,0.63,0.69}{\textbf{\textit{{#1}}}}}
    \newcommand{\CommentVarTok}[1]{\textcolor[rgb]{0.38,0.63,0.69}{\textbf{\textit{{#1}}}}}
    \newcommand{\VariableTok}[1]{\textcolor[rgb]{0.10,0.09,0.49}{{#1}}}
    \newcommand{\ControlFlowTok}[1]{\textcolor[rgb]{0.00,0.44,0.13}{\textbf{{#1}}}}
    \newcommand{\OperatorTok}[1]{\textcolor[rgb]{0.40,0.40,0.40}{{#1}}}
    \newcommand{\BuiltInTok}[1]{{#1}}
    \newcommand{\ExtensionTok}[1]{{#1}}
    \newcommand{\PreprocessorTok}[1]{\textcolor[rgb]{0.74,0.48,0.00}{{#1}}}
    \newcommand{\AttributeTok}[1]{\textcolor[rgb]{0.49,0.56,0.16}{{#1}}}
    \newcommand{\InformationTok}[1]{\textcolor[rgb]{0.38,0.63,0.69}{\textbf{\textit{{#1}}}}}
    \newcommand{\WarningTok}[1]{\textcolor[rgb]{0.38,0.63,0.69}{\textbf{\textit{{#1}}}}}
    
    
    % Define a nice break command that doesn't care if a line doesn't already
    % exist.
    \def\br{\hspace*{\fill} \\* }
    % Math Jax compatibility definitions
    \def\gt{>}
    \def\lt{<}
    \let\Oldtex\TeX
    \let\Oldlatex\LaTeX
    \renewcommand{\TeX}{\textrm{\Oldtex}}
    \renewcommand{\LaTeX}{\textrm{\Oldlatex}}
    % Document parameters
    % Document title
    \title{Manantial Calorífico}
    \author{Jose Ramiro Fuentes Lara}
    \date{8 de Octubre del 2019}

    
    
    
    
% Pygments definitions
\makeatletter
\def\PY@reset{\let\PY@it=\relax \let\PY@bf=\relax%
    \let\PY@ul=\relax \let\PY@tc=\relax%
    \let\PY@bc=\relax \let\PY@ff=\relax}
\def\PY@tok#1{\csname PY@tok@#1\endcsname}
\def\PY@toks#1+{\ifx\relax#1\empty\else%
    \PY@tok{#1}\expandafter\PY@toks\fi}
\def\PY@do#1{\PY@bc{\PY@tc{\PY@ul{%
    \PY@it{\PY@bf{\PY@ff{#1}}}}}}}
\def\PY#1#2{\PY@reset\PY@toks#1+\relax+\PY@do{#2}}

\expandafter\def\csname PY@tok@w\endcsname{\def\PY@tc##1{\textcolor[rgb]{0.73,0.73,0.73}{##1}}}
\expandafter\def\csname PY@tok@c\endcsname{\let\PY@it=\textit\def\PY@tc##1{\textcolor[rgb]{0.25,0.50,0.50}{##1}}}
\expandafter\def\csname PY@tok@cp\endcsname{\def\PY@tc##1{\textcolor[rgb]{0.74,0.48,0.00}{##1}}}
\expandafter\def\csname PY@tok@k\endcsname{\let\PY@bf=\textbf\def\PY@tc##1{\textcolor[rgb]{0.00,0.50,0.00}{##1}}}
\expandafter\def\csname PY@tok@kp\endcsname{\def\PY@tc##1{\textcolor[rgb]{0.00,0.50,0.00}{##1}}}
\expandafter\def\csname PY@tok@kt\endcsname{\def\PY@tc##1{\textcolor[rgb]{0.69,0.00,0.25}{##1}}}
\expandafter\def\csname PY@tok@o\endcsname{\def\PY@tc##1{\textcolor[rgb]{0.40,0.40,0.40}{##1}}}
\expandafter\def\csname PY@tok@ow\endcsname{\let\PY@bf=\textbf\def\PY@tc##1{\textcolor[rgb]{0.67,0.13,1.00}{##1}}}
\expandafter\def\csname PY@tok@nb\endcsname{\def\PY@tc##1{\textcolor[rgb]{0.00,0.50,0.00}{##1}}}
\expandafter\def\csname PY@tok@nf\endcsname{\def\PY@tc##1{\textcolor[rgb]{0.00,0.00,1.00}{##1}}}
\expandafter\def\csname PY@tok@nc\endcsname{\let\PY@bf=\textbf\def\PY@tc##1{\textcolor[rgb]{0.00,0.00,1.00}{##1}}}
\expandafter\def\csname PY@tok@nn\endcsname{\let\PY@bf=\textbf\def\PY@tc##1{\textcolor[rgb]{0.00,0.00,1.00}{##1}}}
\expandafter\def\csname PY@tok@ne\endcsname{\let\PY@bf=\textbf\def\PY@tc##1{\textcolor[rgb]{0.82,0.25,0.23}{##1}}}
\expandafter\def\csname PY@tok@nv\endcsname{\def\PY@tc##1{\textcolor[rgb]{0.10,0.09,0.49}{##1}}}
\expandafter\def\csname PY@tok@no\endcsname{\def\PY@tc##1{\textcolor[rgb]{0.53,0.00,0.00}{##1}}}
\expandafter\def\csname PY@tok@nl\endcsname{\def\PY@tc##1{\textcolor[rgb]{0.63,0.63,0.00}{##1}}}
\expandafter\def\csname PY@tok@ni\endcsname{\let\PY@bf=\textbf\def\PY@tc##1{\textcolor[rgb]{0.60,0.60,0.60}{##1}}}
\expandafter\def\csname PY@tok@na\endcsname{\def\PY@tc##1{\textcolor[rgb]{0.49,0.56,0.16}{##1}}}
\expandafter\def\csname PY@tok@nt\endcsname{\let\PY@bf=\textbf\def\PY@tc##1{\textcolor[rgb]{0.00,0.50,0.00}{##1}}}
\expandafter\def\csname PY@tok@nd\endcsname{\def\PY@tc##1{\textcolor[rgb]{0.67,0.13,1.00}{##1}}}
\expandafter\def\csname PY@tok@s\endcsname{\def\PY@tc##1{\textcolor[rgb]{0.73,0.13,0.13}{##1}}}
\expandafter\def\csname PY@tok@sd\endcsname{\let\PY@it=\textit\def\PY@tc##1{\textcolor[rgb]{0.73,0.13,0.13}{##1}}}
\expandafter\def\csname PY@tok@si\endcsname{\let\PY@bf=\textbf\def\PY@tc##1{\textcolor[rgb]{0.73,0.40,0.53}{##1}}}
\expandafter\def\csname PY@tok@se\endcsname{\let\PY@bf=\textbf\def\PY@tc##1{\textcolor[rgb]{0.73,0.40,0.13}{##1}}}
\expandafter\def\csname PY@tok@sr\endcsname{\def\PY@tc##1{\textcolor[rgb]{0.73,0.40,0.53}{##1}}}
\expandafter\def\csname PY@tok@ss\endcsname{\def\PY@tc##1{\textcolor[rgb]{0.10,0.09,0.49}{##1}}}
\expandafter\def\csname PY@tok@sx\endcsname{\def\PY@tc##1{\textcolor[rgb]{0.00,0.50,0.00}{##1}}}
\expandafter\def\csname PY@tok@m\endcsname{\def\PY@tc##1{\textcolor[rgb]{0.40,0.40,0.40}{##1}}}
\expandafter\def\csname PY@tok@gh\endcsname{\let\PY@bf=\textbf\def\PY@tc##1{\textcolor[rgb]{0.00,0.00,0.50}{##1}}}
\expandafter\def\csname PY@tok@gu\endcsname{\let\PY@bf=\textbf\def\PY@tc##1{\textcolor[rgb]{0.50,0.00,0.50}{##1}}}
\expandafter\def\csname PY@tok@gd\endcsname{\def\PY@tc##1{\textcolor[rgb]{0.63,0.00,0.00}{##1}}}
\expandafter\def\csname PY@tok@gi\endcsname{\def\PY@tc##1{\textcolor[rgb]{0.00,0.63,0.00}{##1}}}
\expandafter\def\csname PY@tok@gr\endcsname{\def\PY@tc##1{\textcolor[rgb]{1.00,0.00,0.00}{##1}}}
\expandafter\def\csname PY@tok@ge\endcsname{\let\PY@it=\textit}
\expandafter\def\csname PY@tok@gs\endcsname{\let\PY@bf=\textbf}
\expandafter\def\csname PY@tok@gp\endcsname{\let\PY@bf=\textbf\def\PY@tc##1{\textcolor[rgb]{0.00,0.00,0.50}{##1}}}
\expandafter\def\csname PY@tok@go\endcsname{\def\PY@tc##1{\textcolor[rgb]{0.53,0.53,0.53}{##1}}}
\expandafter\def\csname PY@tok@gt\endcsname{\def\PY@tc##1{\textcolor[rgb]{0.00,0.27,0.87}{##1}}}
\expandafter\def\csname PY@tok@err\endcsname{\def\PY@bc##1{\setlength{\fboxsep}{0pt}\fcolorbox[rgb]{1.00,0.00,0.00}{1,1,1}{\strut ##1}}}
\expandafter\def\csname PY@tok@kc\endcsname{\let\PY@bf=\textbf\def\PY@tc##1{\textcolor[rgb]{0.00,0.50,0.00}{##1}}}
\expandafter\def\csname PY@tok@kd\endcsname{\let\PY@bf=\textbf\def\PY@tc##1{\textcolor[rgb]{0.00,0.50,0.00}{##1}}}
\expandafter\def\csname PY@tok@kn\endcsname{\let\PY@bf=\textbf\def\PY@tc##1{\textcolor[rgb]{0.00,0.50,0.00}{##1}}}
\expandafter\def\csname PY@tok@kr\endcsname{\let\PY@bf=\textbf\def\PY@tc##1{\textcolor[rgb]{0.00,0.50,0.00}{##1}}}
\expandafter\def\csname PY@tok@bp\endcsname{\def\PY@tc##1{\textcolor[rgb]{0.00,0.50,0.00}{##1}}}
\expandafter\def\csname PY@tok@fm\endcsname{\def\PY@tc##1{\textcolor[rgb]{0.00,0.00,1.00}{##1}}}
\expandafter\def\csname PY@tok@vc\endcsname{\def\PY@tc##1{\textcolor[rgb]{0.10,0.09,0.49}{##1}}}
\expandafter\def\csname PY@tok@vg\endcsname{\def\PY@tc##1{\textcolor[rgb]{0.10,0.09,0.49}{##1}}}
\expandafter\def\csname PY@tok@vi\endcsname{\def\PY@tc##1{\textcolor[rgb]{0.10,0.09,0.49}{##1}}}
\expandafter\def\csname PY@tok@vm\endcsname{\def\PY@tc##1{\textcolor[rgb]{0.10,0.09,0.49}{##1}}}
\expandafter\def\csname PY@tok@sa\endcsname{\def\PY@tc##1{\textcolor[rgb]{0.73,0.13,0.13}{##1}}}
\expandafter\def\csname PY@tok@sb\endcsname{\def\PY@tc##1{\textcolor[rgb]{0.73,0.13,0.13}{##1}}}
\expandafter\def\csname PY@tok@sc\endcsname{\def\PY@tc##1{\textcolor[rgb]{0.73,0.13,0.13}{##1}}}
\expandafter\def\csname PY@tok@dl\endcsname{\def\PY@tc##1{\textcolor[rgb]{0.73,0.13,0.13}{##1}}}
\expandafter\def\csname PY@tok@s2\endcsname{\def\PY@tc##1{\textcolor[rgb]{0.73,0.13,0.13}{##1}}}
\expandafter\def\csname PY@tok@sh\endcsname{\def\PY@tc##1{\textcolor[rgb]{0.73,0.13,0.13}{##1}}}
\expandafter\def\csname PY@tok@s1\endcsname{\def\PY@tc##1{\textcolor[rgb]{0.73,0.13,0.13}{##1}}}
\expandafter\def\csname PY@tok@mb\endcsname{\def\PY@tc##1{\textcolor[rgb]{0.40,0.40,0.40}{##1}}}
\expandafter\def\csname PY@tok@mf\endcsname{\def\PY@tc##1{\textcolor[rgb]{0.40,0.40,0.40}{##1}}}
\expandafter\def\csname PY@tok@mh\endcsname{\def\PY@tc##1{\textcolor[rgb]{0.40,0.40,0.40}{##1}}}
\expandafter\def\csname PY@tok@mi\endcsname{\def\PY@tc##1{\textcolor[rgb]{0.40,0.40,0.40}{##1}}}
\expandafter\def\csname PY@tok@il\endcsname{\def\PY@tc##1{\textcolor[rgb]{0.40,0.40,0.40}{##1}}}
\expandafter\def\csname PY@tok@mo\endcsname{\def\PY@tc##1{\textcolor[rgb]{0.40,0.40,0.40}{##1}}}
\expandafter\def\csname PY@tok@ch\endcsname{\let\PY@it=\textit\def\PY@tc##1{\textcolor[rgb]{0.25,0.50,0.50}{##1}}}
\expandafter\def\csname PY@tok@cm\endcsname{\let\PY@it=\textit\def\PY@tc##1{\textcolor[rgb]{0.25,0.50,0.50}{##1}}}
\expandafter\def\csname PY@tok@cpf\endcsname{\let\PY@it=\textit\def\PY@tc##1{\textcolor[rgb]{0.25,0.50,0.50}{##1}}}
\expandafter\def\csname PY@tok@c1\endcsname{\let\PY@it=\textit\def\PY@tc##1{\textcolor[rgb]{0.25,0.50,0.50}{##1}}}
\expandafter\def\csname PY@tok@cs\endcsname{\let\PY@it=\textit\def\PY@tc##1{\textcolor[rgb]{0.25,0.50,0.50}{##1}}}

\def\PYZbs{\char`\\}
\def\PYZus{\char`\_}
\def\PYZob{\char`\{}
\def\PYZcb{\char`\}}
\def\PYZca{\char`\^}
\def\PYZam{\char`\&}
\def\PYZlt{\char`\<}
\def\PYZgt{\char`\>}
\def\PYZsh{\char`\#}
\def\PYZpc{\char`\%}
\def\PYZdl{\char`\$}
\def\PYZhy{\char`\-}
\def\PYZsq{\char`\'}
\def\PYZdq{\char`\"}
\def\PYZti{\char`\~}
% for compatibility with earlier versions
\def\PYZat{@}
\def\PYZlb{[}
\def\PYZrb{]}
\makeatother


    % For linebreaks inside Verbatim environment from package fancyvrb. 
    \makeatletter
        \newbox\Wrappedcontinuationbox 
        \newbox\Wrappedvisiblespacebox 
        \newcommand*\Wrappedvisiblespace {\textcolor{red}{\textvisiblespace}} 
        \newcommand*\Wrappedcontinuationsymbol {\textcolor{red}{\llap{\tiny$\m@th\hookrightarrow$}}} 
        \newcommand*\Wrappedcontinuationindent {3ex } 
        \newcommand*\Wrappedafterbreak {\kern\Wrappedcontinuationindent\copy\Wrappedcontinuationbox} 
        % Take advantage of the already applied Pygments mark-up to insert 
        % potential linebreaks for TeX processing. 
        %        {, <, #, %, $, ' and ": go to next line. 
        %        _, }, ^, &, >, - and ~: stay at end of broken line. 
        % Use of \textquotesingle for straight quote. 
        \newcommand*\Wrappedbreaksatspecials {% 
            \def\PYGZus{\discretionary{\char`\_}{\Wrappedafterbreak}{\char`\_}}% 
            \def\PYGZob{\discretionary{}{\Wrappedafterbreak\char`\{}{\char`\{}}% 
            \def\PYGZcb{\discretionary{\char`\}}{\Wrappedafterbreak}{\char`\}}}% 
            \def\PYGZca{\discretionary{\char`\^}{\Wrappedafterbreak}{\char`\^}}% 
            \def\PYGZam{\discretionary{\char`\&}{\Wrappedafterbreak}{\char`\&}}% 
            \def\PYGZlt{\discretionary{}{\Wrappedafterbreak\char`\<}{\char`\<}}% 
            \def\PYGZgt{\discretionary{\char`\>}{\Wrappedafterbreak}{\char`\>}}% 
            \def\PYGZsh{\discretionary{}{\Wrappedafterbreak\char`\#}{\char`\#}}% 
            \def\PYGZpc{\discretionary{}{\Wrappedafterbreak\char`\%}{\char`\%}}% 
            \def\PYGZdl{\discretionary{}{\Wrappedafterbreak\char`\$}{\char`\$}}% 
            \def\PYGZhy{\discretionary{\char`\-}{\Wrappedafterbreak}{\char`\-}}% 
            \def\PYGZsq{\discretionary{}{\Wrappedafterbreak\textquotesingle}{\textquotesingle}}% 
            \def\PYGZdq{\discretionary{}{\Wrappedafterbreak\char`\"}{\char`\"}}% 
            \def\PYGZti{\discretionary{\char`\~}{\Wrappedafterbreak}{\char`\~}}% 
        } 
        % Some characters . , ; ? ! / are not pygmentized. 
        % This macro makes them "active" and they will insert potential linebreaks 
        \newcommand*\Wrappedbreaksatpunct {% 
            \lccode`\~`\.\lowercase{\def~}{\discretionary{\hbox{\char`\.}}{\Wrappedafterbreak}{\hbox{\char`\.}}}% 
            \lccode`\~`\,\lowercase{\def~}{\discretionary{\hbox{\char`\,}}{\Wrappedafterbreak}{\hbox{\char`\,}}}% 
            \lccode`\~`\;\lowercase{\def~}{\discretionary{\hbox{\char`\;}}{\Wrappedafterbreak}{\hbox{\char`\;}}}% 
            \lccode`\~`\:\lowercase{\def~}{\discretionary{\hbox{\char`\:}}{\Wrappedafterbreak}{\hbox{\char`\:}}}% 
            \lccode`\~`\?\lowercase{\def~}{\discretionary{\hbox{\char`\?}}{\Wrappedafterbreak}{\hbox{\char`\?}}}% 
            \lccode`\~`\!\lowercase{\def~}{\discretionary{\hbox{\char`\!}}{\Wrappedafterbreak}{\hbox{\char`\!}}}% 
            \lccode`\~`\/\lowercase{\def~}{\discretionary{\hbox{\char`\/}}{\Wrappedafterbreak}{\hbox{\char`\/}}}% 
            \catcode`\.\active
            \catcode`\,\active 
            \catcode`\;\active
            \catcode`\:\active
            \catcode`\?\active
            \catcode`\!\active
            \catcode`\/\active 
            \lccode`\~`\~ 	
        }
    \makeatother

    \let\OriginalVerbatim=\Verbatim
    \makeatletter
    \renewcommand{\Verbatim}[1][1]{%
        %\parskip\z@skip
        \sbox\Wrappedcontinuationbox {\Wrappedcontinuationsymbol}%
        \sbox\Wrappedvisiblespacebox {\FV@SetupFont\Wrappedvisiblespace}%
        \def\FancyVerbFormatLine ##1{\hsize\linewidth
            \vtop{\raggedright\hyphenpenalty\z@\exhyphenpenalty\z@
                \doublehyphendemerits\z@\finalhyphendemerits\z@
                \strut ##1\strut}%
        }%
        % If the linebreak is at a space, the latter will be displayed as visible
        % space at end of first line, and a continuation symbol starts next line.
        % Stretch/shrink are however usually zero for typewriter font.
        \def\FV@Space {%
            \nobreak\hskip\z@ plus\fontdimen3\font minus\fontdimen4\font
            \discretionary{\copy\Wrappedvisiblespacebox}{\Wrappedafterbreak}
            {\kern\fontdimen2\font}%
        }%
        
        % Allow breaks at special characters using \PYG... macros.
        \Wrappedbreaksatspecials
        % Breaks at punctuation characters . , ; ? ! and / need catcode=\active 	
        \OriginalVerbatim[#1,codes*=\Wrappedbreaksatpunct]%
    }
    \makeatother

    % Exact colors from NB
    \definecolor{incolor}{HTML}{303F9F}
    \definecolor{outcolor}{HTML}{D84315}
    \definecolor{cellborder}{HTML}{CFCFCF}
    \definecolor{cellbackground}{HTML}{F7F7F7}
    
    % prompt
    \newcommand{\prompt}[4]{
        \llap{{\color{#2}[#3]: #4}}\vspace{-1.25em}
    }
    

    
    % Prevent overflowing lines due to hard-to-break entities
    \sloppy 
    % Setup hyperref package
    \hypersetup{
      breaklinks=true,  % so long urls are correctly broken across lines
      colorlinks=true,
      urlcolor=urlcolor,
      linkcolor=linkcolor,
      citecolor=citecolor,
      }
    % Slightly bigger margins than the latex defaults
    
    \geometry{verbose,tmargin=1in,bmargin=1in,lmargin=1in,rmargin=1in}
    
    

    \begin{document}
    
    
    \maketitle
    
    

    
    \hypertarget{conducciuxf3n-de-calor-con-un-manantial-caloruxedfico-de-origen-nuclear}{%
\section{Conducción de calor con un manantial calorífico de origen
nuclear}\label{conducciuxf3n-de-calor-con-un-manantial-caloruxedfico-de-origen-nuclear}}

Considera un elemento de combustible nuclear de forma esférica donde el
sistema consta de una esfera de material fisionable de radio R\^{}F
revestido de una cubierta esférica de aluminio con un radio externo de
R\^{}c. El manantial de energía calorífica que resulta de la fisión
nuclear Sn puede representarse con la función parabolica:\\

\(S_n=S_{n0}[1-b(\frac{r}{R^F})^2]\) \\

Donde:

\begin{itemize}
\tightlist
\item
  \(Sn_0\) = velocidad volumétrica de producción de calor en el centro
  de la esféra
\item
  \(b\) = Constante admiensional que toma volumenes entre 0 y 1
\end{itemize}

    \hypertarget{perfiles-obtenidos-en-la-soluciuxf3n-de-forma-analuxedtica}{%
\subsection{Perfiles obtenidos en la solución de forma
analítica}\label{perfiles-obtenidos-en-la-soluciuxf3n-de-forma-analuxedtica}}

Perfil de distribución en el material fisionable:\\

\(T^f={T_O+\frac{(R^{F})^3}{K_c}Sn_0(\frac{1}{3}-\frac{b}{5})(\frac{1}{R^F}-\frac{1}{R^c})}+\frac{Sn_0}{K^F}[-\frac{r^2}{6}+\frac{b}{20(R^F)^2}r^4 +\frac{(R^F)^2}{6}-\frac{(R^F)^2b}{20}]\)\\

Perfil de distribución de la cubierta:\\

\(T^C=T_O+\frac{R^{F3}}{K_c}Sn_0(\frac{1}{3}-\frac{b}{5})(\frac{1}{r}-\frac{1}{R^c})\)

    \hypertarget{tareas-a-determinar-en-el-ejercicio}{%
\subsubsection{Tareas a determinar en el
ejercicio}\label{tareas-a-determinar-en-el-ejercicio}}

\textbf{a)} Reproducir los perfiles de Temperatura de ambas esféras para
distintos valores \(b\) ={[}0:.1:1{]} considerando los valores
perinentes contantes de los demás parámetros. en base a estos perfiles
decir que significado físico puede tener el parámetro b y a que
propiedad física del material fisionable pudiera estar relacionada con
dicho parámetro.

Notas personales: - Límite g de calor Sn - Fundición del aluminio\\

\textbf{b)} Analizar el efecto de cada parámetro * \(Sn_0\) (Generación
de calor en el centro de la esfera) * \(R^F\) (Radio del elemento
fisionable) * \(R^C\) (Radio de la cubierta de aluminio) * \(T_0\)
(Temperatura en la cubierta) * \(K^F\) (Conductividad del material
fisionable)

Notas personales: - El \(K^C\) del aluminio es una propiedad conocida

\textbf{c)} En base a los gráficos obtenidos decir bajo que condiciones
sería seguro tener el material fisionable operando sin que este ni la
cubierta de aluminio lleguen a fundirse , es decir, que siempre se
mantengan operando es estado físico solido.

Nota del profesor: * Utilizar como mínimo 3 materiales
Uranio,Plutonio,Torio(Buscar en bibliografía)

    \hypertarget{aspectos-a-tomar-en-cuenta}{%
\subsection{Aspectos a tomar en
cuenta}\label{aspectos-a-tomar-en-cuenta}}

Puntos de fusión:

\begin{itemize}
\tightlist
\item
  \(PF_{Aluminio}\) = 934.47 k
\item
  \(PF_{Uranio}\) = 1405.15 k
\item
  \(PF_{Plutonio}\) = 912.55 k
\item
  \(PF_{Torio}\) = 2023.15 k
\end{itemize}

Conductividad termica: \\\\
\(K_{Aluminio}\) = 205.0\(\frac{W}{mK}\)\\
\(K_{Uranio}\) = 27.6 \(\frac{W}{mK}\) \\
\(K_{Plutonio}\) = 6.74\(\frac{W}{mK}\) \\
 \(K_{Torio}\) = 54.0 \(\frac{W}{mK}\)\\

    \hypertarget{valores-considerados}{%
\subsection{Valores considerados}\label{valores-considerados}}

\(Sn_0=1*10^6\)

    \hypertarget{cuxf3digo}{%
\section{Código}\label{cuxf3digo}}

    \begin{tcolorbox}[breakable, size=fbox, boxrule=1pt, pad at break*=1mm,colback=cellbackground, colframe=cellborder]
\prompt{In}{incolor}{2}{\hspace{4pt}}
\begin{Verbatim}[commandchars=\\\{\}]
\PY{k+kn}{import} \PY{n+nn}{numpy} \PY{k}{as} \PY{n+nn}{np}
\PY{k+kn}{import} \PY{n+nn}{matplotlib}\PY{n+nn}{.}\PY{n+nn}{pyplot} \PY{k}{as} \PY{n+nn}{plt}
\PY{k+kn}{from} \PY{n+nn}{math} \PY{k}{import} \PY{n}{fabs}
\end{Verbatim}
\end{tcolorbox}

    \begin{tcolorbox}[breakable, size=fbox, boxrule=1pt, pad at break*=1mm,colback=cellbackground, colframe=cellborder]
\prompt{In}{incolor}{305}{\hspace{4pt}}
\begin{Verbatim}[commandchars=\\\{\}]
\PY{c+c1}{\PYZsh{} Condiciones del ambiente}
\PY{n}{T0} \PY{o}{=} \PY{l+m+mf}{863.15} \PY{c+c1}{\PYZsh{} Temperatura mas común del refrigerante sin considerar capa limite termica}
\PY{n}{Rf} \PY{o}{=} \PY{l+m+mf}{0.1} \PY{c+c1}{\PYZsh{}.25 \PYZsh{} Radio del material fisionable, medio metro}
\PY{n}{Rc} \PY{o}{=} \PY{l+m+mi}{1} \PY{c+c1}{\PYZsh{} Radio del material de cubierta}

\PY{n}{Aluminio} \PY{o}{=} \PY{p}{\PYZob{}}
    \PY{l+s+s1}{\PYZsq{}}\PY{l+s+s1}{Symbol}\PY{l+s+s1}{\PYZsq{}}\PY{p}{:} \PY{l+s+s2}{\PYZdq{}}\PY{l+s+s2}{Al}\PY{l+s+s2}{\PYZdq{}}\PY{p}{,}
    \PY{l+s+s1}{\PYZsq{}}\PY{l+s+s1}{PF}\PY{l+s+s1}{\PYZsq{}}\PY{p}{:}\PY{l+m+mf}{934.47}\PY{p}{,}
    \PY{l+s+s1}{\PYZsq{}}\PY{l+s+s1}{K}\PY{l+s+s1}{\PYZsq{}}\PY{p}{:}\PY{l+m+mf}{205.0}\PY{p}{,}
\PY{p}{\PYZcb{}}

\PY{n}{Torio} \PY{o}{=} \PY{p}{\PYZob{}}
    \PY{l+s+s1}{\PYZsq{}}\PY{l+s+s1}{Symbol}\PY{l+s+s1}{\PYZsq{}}\PY{p}{:} \PY{l+s+s2}{\PYZdq{}}\PY{l+s+s2}{Th}\PY{l+s+s2}{\PYZdq{}}\PY{p}{,}
    \PY{l+s+s1}{\PYZsq{}}\PY{l+s+s1}{PF}\PY{l+s+s1}{\PYZsq{}} \PY{p}{:} \PY{l+m+mf}{2023.15}\PY{p}{,}
    \PY{l+s+s1}{\PYZsq{}}\PY{l+s+s1}{K}\PY{l+s+s1}{\PYZsq{}} \PY{p}{:} \PY{l+m+mf}{54.0}\PY{p}{,}
    \PY{l+s+s1}{\PYZsq{}}\PY{l+s+s1}{Sn0}\PY{l+s+s1}{\PYZsq{}} \PY{p}{:} \PY{l+m+mf}{161.4}\PY{o}{*}\PY{l+m+mi}{10}\PY{o}{*}\PY{o}{*}\PY{l+m+mi}{6}
\PY{p}{\PYZcb{}}

\PY{n}{Uranio} \PY{o}{=} \PY{p}{\PYZob{}}
    \PY{l+s+s1}{\PYZsq{}}\PY{l+s+s1}{Symbol}\PY{l+s+s1}{\PYZsq{}}\PY{p}{:} \PY{l+s+s2}{\PYZdq{}}\PY{l+s+s2}{U}\PY{l+s+s2}{\PYZdq{}}\PY{p}{,}
    \PY{l+s+s1}{\PYZsq{}}\PY{l+s+s1}{PF}\PY{l+s+s1}{\PYZsq{}} \PY{p}{:} \PY{l+m+mf}{1405.15}\PY{p}{,}
    \PY{l+s+s1}{\PYZsq{}}\PY{l+s+s1}{K}\PY{l+s+s1}{\PYZsq{}} \PY{p}{:} \PY{l+m+mf}{27.6}\PY{p}{,} 
    \PY{l+s+s1}{\PYZsq{}}\PY{l+s+s1}{Sn0}\PY{l+s+s1}{\PYZsq{}} \PY{p}{:} \PY{l+m+mi}{2}\PY{o}{*}\PY{l+m+mi}{10}\PY{o}{*}\PY{o}{*}\PY{l+m+mi}{8}
\PY{p}{\PYZcb{}}

\PY{n}{Plutonio} \PY{o}{=} \PY{p}{\PYZob{}}
    \PY{l+s+s1}{\PYZsq{}}\PY{l+s+s1}{Symbol}\PY{l+s+s1}{\PYZsq{}}\PY{p}{:} \PY{l+s+s2}{\PYZdq{}}\PY{l+s+s2}{Pu}\PY{l+s+s2}{\PYZdq{}}\PY{p}{,}
    \PY{l+s+s1}{\PYZsq{}}\PY{l+s+s1}{PF}\PY{l+s+s1}{\PYZsq{}} \PY{p}{:} \PY{l+m+mf}{912.55}\PY{p}{,}
    \PY{l+s+s1}{\PYZsq{}}\PY{l+s+s1}{K}\PY{l+s+s1}{\PYZsq{}} \PY{p}{:} \PY{l+m+mf}{6.74}\PY{p}{,}
    \PY{l+s+s1}{\PYZsq{}}\PY{l+s+s1}{Sn0}\PY{l+s+s1}{\PYZsq{}} \PY{p}{:} \PY{l+m+mf}{11.1}\PY{o}{*}\PY{l+m+mi}{10}\PY{o}{*}\PY{o}{*}\PY{l+m+mi}{6}
\PY{p}{\PYZcb{}}
\end{Verbatim}
\end{tcolorbox}

    \begin{tcolorbox}[breakable, size=fbox, boxrule=1pt, pad at break*=1mm,colback=cellbackground, colframe=cellborder]
\prompt{In}{incolor}{306}{\hspace{4pt}}
\begin{Verbatim}[commandchars=\\\{\}]
\PY{k}{def} \PY{n+nf}{tempFisionable}\PY{p}{(}\PY{n}{T0}\PY{p}{,}\PY{n}{Rf}\PY{p}{,}\PY{n}{Rc}\PY{p}{,}\PY{n}{r}\PY{p}{,}\PY{n}{b}\PY{p}{,}\PY{n}{Fisionable}\PY{p}{,}\PY{n}{Cubierta}\PY{p}{)}\PY{p}{:}
    
    \PY{n}{A1} \PY{o}{=} \PY{p}{(}\PY{p}{(}\PY{n}{Rf}\PY{o}{*}\PY{o}{*}\PY{l+m+mi}{3}\PY{p}{)}\PY{o}{/}\PY{n}{Cubierta}\PY{p}{[}\PY{l+s+s1}{\PYZsq{}}\PY{l+s+s1}{K}\PY{l+s+s1}{\PYZsq{}}\PY{p}{]}\PY{p}{)} \PY{o}{*} \PY{n}{Fisionable}\PY{p}{[}\PY{l+s+s1}{\PYZsq{}}\PY{l+s+s1}{Sn0}\PY{l+s+s1}{\PYZsq{}}\PY{p}{]} \PY{o}{*} \PY{p}{(}\PY{p}{(}\PY{l+m+mi}{1}\PY{o}{/}\PY{l+m+mi}{3}\PY{p}{)}\PY{o}{\PYZhy{}}\PY{p}{(}\PY{n}{b}\PY{o}{/}\PY{l+m+mi}{5}\PY{p}{)}\PY{p}{)} \PY{o}{*} \PY{p}{(}\PY{p}{(}\PY{l+m+mi}{1}\PY{o}{/}\PY{n}{Rf}\PY{p}{)}\PY{o}{\PYZhy{}}\PY{p}{(}\PY{l+m+mi}{1}\PY{o}{/}\PY{n}{Rc}\PY{p}{)}\PY{p}{)}
    \PY{n}{A2} \PY{o}{=} \PY{p}{(}\PY{n}{Fisionable}\PY{p}{[}\PY{l+s+s1}{\PYZsq{}}\PY{l+s+s1}{Sn0}\PY{l+s+s1}{\PYZsq{}}\PY{p}{]}\PY{o}{/}\PY{n}{Fisionable}\PY{p}{[}\PY{l+s+s1}{\PYZsq{}}\PY{l+s+s1}{K}\PY{l+s+s1}{\PYZsq{}}\PY{p}{]}\PY{p}{)} \PY{o}{*} \PY{p}{(} \PY{p}{(}\PY{o}{\PYZhy{}}\PY{p}{(}\PY{n}{r}\PY{o}{*}\PY{o}{*}\PY{l+m+mi}{2}\PY{p}{)}\PY{o}{/}\PY{l+m+mi}{6}\PY{p}{)} \PY{o}{+} \PY{p}{(} \PY{p}{(} \PY{n}{b}\PY{o}{*}\PY{p}{(}\PY{n}{r}\PY{o}{*}\PY{o}{*}\PY{l+m+mi}{4}\PY{p}{)} \PY{p}{)}\PY{o}{/}\PY{p}{(} \PY{l+m+mi}{20}\PY{o}{*}\PY{p}{(}\PY{n}{Rf}\PY{o}{*}\PY{o}{*}\PY{l+m+mi}{2} \PY{p}{)}\PY{p}{)} \PY{p}{)} \PY{o}{+} \PY{p}{(}\PY{p}{(}\PY{n}{Rf}\PY{o}{*}\PY{o}{*}\PY{l+m+mi}{2}\PY{p}{)}\PY{o}{/}\PY{l+m+mi}{6}\PY{p}{)} \PY{o}{\PYZhy{}} \PY{p}{(}\PY{p}{(}\PY{n}{Rf}\PY{o}{*}\PY{o}{*}\PY{l+m+mi}{2}\PY{p}{)}\PY{o}{*}\PY{n}{b}\PY{p}{)}\PY{o}{/}\PY{l+m+mi}{20}\PY{p}{)}
    \PY{n}{Tf} \PY{o}{=} \PY{n}{T0} \PY{o}{+} \PY{n}{A1} \PY{o}{+} \PY{n}{A2}
    
    \PY{k}{return} \PY{n}{Tf}

\PY{k}{def} \PY{n+nf}{tempCubierta}\PY{p}{(}\PY{n}{T0}\PY{p}{,}\PY{n}{Rf}\PY{p}{,}\PY{n}{Rc}\PY{p}{,}\PY{n}{r}\PY{p}{,}\PY{n}{b}\PY{p}{,}\PY{n}{Fisionable}\PY{p}{,}\PY{n}{Cubierta}\PY{p}{)}\PY{p}{:}
    \PY{n}{Tc} \PY{o}{=} \PY{n}{T0} \PY{o}{+} \PY{p}{(}\PY{p}{(}\PY{n}{Rf}\PY{o}{*}\PY{o}{*}\PY{l+m+mi}{3}\PY{p}{)}\PY{o}{/}\PY{n}{Cubierta}\PY{p}{[}\PY{l+s+s1}{\PYZsq{}}\PY{l+s+s1}{K}\PY{l+s+s1}{\PYZsq{}}\PY{p}{]}\PY{p}{)} \PY{o}{*} \PY{n}{Fisionable}\PY{p}{[}\PY{l+s+s1}{\PYZsq{}}\PY{l+s+s1}{Sn0}\PY{l+s+s1}{\PYZsq{}}\PY{p}{]} \PY{o}{*} \PY{p}{(}\PY{p}{(}\PY{l+m+mi}{1}\PY{o}{/}\PY{l+m+mi}{3}\PY{p}{)}\PY{o}{\PYZhy{}}\PY{p}{(}\PY{n}{b}\PY{o}{/}\PY{l+m+mi}{5}\PY{p}{)}\PY{p}{)} \PY{o}{*} \PY{p}{(}\PY{p}{(}\PY{l+m+mi}{1}\PY{o}{/}\PY{n}{r}\PY{p}{)}\PY{o}{\PYZhy{}}\PY{p}{(}\PY{l+m+mi}{1}\PY{o}{/}\PY{n}{Rc}\PY{p}{)}\PY{p}{)}
    \PY{k}{return} \PY{n}{Tc}

\PY{k}{def} \PY{n+nf}{analisisSistema}\PY{p}{(}\PY{n}{T0}\PY{p}{,}\PY{n}{Rf}\PY{p}{,}\PY{n}{Rc}\PY{p}{,}\PY{n}{Fisionable}\PY{p}{,}\PY{n}{Cubierta}\PY{p}{)}\PY{p}{:}
    \PY{n}{t} \PY{o}{=} \PY{l+m+mf}{0.001}
    \PY{n}{r} \PY{o}{=} \PY{n}{np}\PY{o}{.}\PY{n}{arange}\PY{p}{(}\PY{l+m+mi}{0}\PY{p}{,}\PY{n}{Rf}\PY{o}{+}\PY{n}{Rc}\PY{o}{+}\PY{n}{t}\PY{p}{,}\PY{n}{t}\PY{p}{)}
    \PY{n}{b} \PY{o}{=} \PY{n}{np}\PY{o}{.}\PY{n}{arange}\PY{p}{(}\PY{l+m+mi}{0}\PY{p}{,}\PY{l+m+mf}{1.1}\PY{p}{,}\PY{o}{.}\PY{l+m+mi}{1}\PY{p}{)}
    
    \PY{k}{for} \PY{n}{Cb} \PY{o+ow}{in} \PY{n}{b}\PY{p}{:}
        \PY{n}{TS}\PY{o}{=}\PY{p}{[}\PY{p}{]}
        \PY{k}{for} \PY{n}{Cr} \PY{o+ow}{in} \PY{n}{r}\PY{p}{:}
            \PY{k}{if} \PY{n}{Cr} \PY{o}{\PYZlt{}}\PY{o}{=} \PY{n}{Rf}\PY{p}{:}
                \PY{n}{TS}\PY{o}{.}\PY{n}{append}\PY{p}{(}\PY{n}{tempFisionable}\PY{p}{(}\PY{n}{T0}\PY{p}{,}\PY{n}{Rf}\PY{p}{,}\PY{n}{Rc}\PY{p}{,}\PY{n}{Cr}\PY{p}{,}\PY{n}{Cb}\PY{p}{,}\PY{n}{Fisionable}\PY{p}{,}\PY{n}{Cubierta}\PY{p}{)}\PY{p}{)}
            \PY{k}{else}\PY{p}{:}
                \PY{n}{TS}\PY{o}{.}\PY{n}{append}\PY{p}{(}\PY{n}{tempCubierta}\PY{p}{(}\PY{n}{T0}\PY{p}{,}\PY{n}{Rf}\PY{p}{,}\PY{n}{Rc}\PY{p}{,}\PY{n}{Cr}\PY{p}{,}\PY{n}{Cb}\PY{p}{,}\PY{n}{Fisionable}\PY{p}{,}\PY{n}{Cubierta}\PY{p}{)}\PY{p}{)}
                
        \PY{n}{plt}\PY{o}{.}\PY{n}{plot}\PY{p}{(}\PY{n}{r}\PY{p}{,}\PY{n}{TS}\PY{p}{,}\PY{l+s+s1}{\PYZsq{}}\PY{l+s+s1}{\PYZhy{}}\PY{l+s+s1}{\PYZsq{}}\PY{p}{,}\PY{n}{label}\PY{o}{=}\PY{n+nb}{round}\PY{p}{(}\PY{n}{Cb}\PY{p}{,}\PY{l+m+mi}{2}\PY{p}{)}\PY{p}{)}
        \PY{n}{plt}\PY{o}{.}\PY{n}{axvline}\PY{p}{(}\PY{n}{x}\PY{o}{=}\PY{n}{Rf}\PY{p}{,}\PY{n}{ls}\PY{o}{=}\PY{l+s+s2}{\PYZdq{}}\PY{l+s+s2}{dotted}\PY{l+s+s2}{\PYZdq{}}\PY{p}{,}\PY{n}{color} \PY{o}{=} \PY{l+s+s1}{\PYZsq{}}\PY{l+s+s1}{blue}\PY{l+s+s1}{\PYZsq{}}\PY{p}{)}
        \PY{n}{plt}\PY{o}{.}\PY{n}{axhline}\PY{p}{(}\PY{n}{y}\PY{o}{=}\PY{n}{T0}\PY{p}{,}\PY{n}{ls}\PY{o}{=}\PY{l+s+s2}{\PYZdq{}}\PY{l+s+s2}{dotted}\PY{l+s+s2}{\PYZdq{}}\PY{p}{,}\PY{n}{color} \PY{o}{=} \PY{l+s+s1}{\PYZsq{}}\PY{l+s+s1}{green}\PY{l+s+s1}{\PYZsq{}}\PY{p}{)}
        \PY{n}{plt}\PY{o}{.}\PY{n}{axhline}\PY{p}{(}\PY{n}{y}\PY{o}{=}\PY{n}{Fisionable}\PY{p}{[}\PY{l+s+s1}{\PYZsq{}}\PY{l+s+s1}{PF}\PY{l+s+s1}{\PYZsq{}}\PY{p}{]}\PY{p}{,}\PY{n}{ls}\PY{o}{=}\PY{l+s+s2}{\PYZdq{}}\PY{l+s+s2}{dotted}\PY{l+s+s2}{\PYZdq{}}\PY{p}{,}\PY{n}{color} \PY{o}{=} \PY{l+s+s1}{\PYZsq{}}\PY{l+s+s1}{red}\PY{l+s+s1}{\PYZsq{}}\PY{p}{)}
        \PY{n}{plt}\PY{o}{.}\PY{n}{axhline}\PY{p}{(}\PY{n}{y}\PY{o}{=}\PY{n}{Cubierta}\PY{p}{[}\PY{l+s+s1}{\PYZsq{}}\PY{l+s+s1}{PF}\PY{l+s+s1}{\PYZsq{}}\PY{p}{]}\PY{p}{,}\PY{n}{ls}\PY{o}{=}\PY{l+s+s2}{\PYZdq{}}\PY{l+s+s2}{dotted}\PY{l+s+s2}{\PYZdq{}}\PY{p}{,}\PY{n}{color} \PY{o}{=} \PY{l+s+s1}{\PYZsq{}}\PY{l+s+s1}{orange}\PY{l+s+s1}{\PYZsq{}}\PY{p}{)}
        \PY{n}{plt}\PY{o}{.}\PY{n}{title}\PY{p}{(}\PY{n}{f}\PY{l+s+s2}{\PYZdq{}}\PY{l+s+s2}{Variación del parámetro B}\PY{l+s+se}{\PYZbs{}n}\PY{l+s+s2}{para el }\PY{l+s+si}{\PYZob{}Fisionable[\PYZsq{}Symbol\PYZsq{}]\PYZcb{}}\PY{l+s+s2}{\PYZdq{}}\PY{p}{)}
        \PY{n}{plt}\PY{o}{.}\PY{n}{grid}\PY{p}{(}\PY{p}{)}
        \PY{n}{plt}\PY{o}{.}\PY{n}{xlabel}\PY{p}{(}\PY{l+s+s1}{\PYZsq{}}\PY{l+s+s1}{Radio (m)}\PY{l+s+s1}{\PYZsq{}}\PY{p}{)}
        \PY{n}{plt}\PY{o}{.}\PY{n}{ylabel}\PY{p}{(}\PY{l+s+s1}{\PYZsq{}}\PY{l+s+s1}{Temperatura (k)}\PY{l+s+s1}{\PYZsq{}}\PY{p}{)}
        \PY{n}{plt}\PY{o}{.}\PY{n}{legend}\PY{p}{(}\PY{p}{)}
    \PY{n}{plt}\PY{o}{.}\PY{n}{show}\PY{p}{(}\PY{p}{)} 
    \PY{k}{pass}    
\end{Verbatim}
\end{tcolorbox}

    \hypertarget{a-perfiles-de-temperatura}{%
\subsection{a) Perfiles de
Temperatura}\label{a-perfiles-de-temperatura}}

Para esta simulación se trataron condiciones en las cuales no se
superaran los puntos de fusión del material fisionable dentro de su
radio o del material de cubierta igualmente. Se planteo la siguiente
notación de lineas punteadas:

\begin{itemize}
\tightlist
\item
  \textbf{Amarilla:} La linea amarilla representa el Punto de fusión del
  material de la cubierta.
\item
  \textbf{Roja:} La linea roja marca el punto de fusión del material
  fisionable.
\item
  \textbf{Verde:} Representa la temperatura de la cubierta o la
  temperatura en equilibrio fuera del sistma.
\item
  \textbf{Azul:} Interfáz entre el radio del fisionable y el radio de
  cubierta.
\end{itemize}

Así se obtuvieron valores de referencia que serán usados mas tarde.

    \begin{tcolorbox}[breakable, size=fbox, boxrule=1pt, pad at break*=1mm,colback=cellbackground, colframe=cellborder]
\prompt{In}{incolor}{335}{\hspace{4pt}}
\begin{Verbatim}[commandchars=\\\{\}]
\PY{n}{analisisSistema}\PY{p}{(}\PY{n}{T0}\PY{p}{,}\PY{o}{.}\PY{l+m+mi}{013}\PY{p}{,}\PY{o}{.}\PY{l+m+mi}{08}\PY{p}{,}\PY{n}{Plutonio}\PY{p}{,}\PY{n}{Aluminio}\PY{p}{)}
\PY{n}{analisisSistema}\PY{p}{(}\PY{n}{T0}\PY{p}{,}\PY{o}{.}\PY{l+m+mi}{0274}\PY{p}{,}\PY{o}{.}\PY{l+m+mi}{043}\PY{p}{,}\PY{n}{Torio}\PY{p}{,}\PY{n}{Aluminio}\PY{p}{)}
\PY{n}{analisisSistema}\PY{p}{(}\PY{n}{T0}\PY{p}{,}\PY{o}{.}\PY{l+m+mi}{017}\PY{p}{,}\PY{o}{.}\PY{l+m+mi}{1}\PY{p}{,}\PY{n}{Uranio}\PY{p}{,}\PY{n}{Aluminio}\PY{p}{)}
\end{Verbatim}
\end{tcolorbox}

    \begin{center}
    \adjustimage{max size={0.9\linewidth}{0.9\paperheight}}{output_10_0.png}
    \end{center}
    { \hspace*{\fill} \\}
    
    \begin{center}
    \adjustimage{max size={0.9\linewidth}{0.9\paperheight}}{output_10_1.png}
    \end{center}
    { \hspace*{\fill} \\}
    
    \begin{center}
    \adjustimage{max size={0.9\linewidth}{0.9\paperheight}}{output_10_2.png}
    \end{center}
    { \hspace*{\fill} \\}
    
    \hypertarget{que-significado-fuxedsico}{%
\subsection{Que significado físico}\label{que-significado-fuxedsico}}

Puede tener el parámetro b y a que propiedad física del material
fisionable pudiera estar relacionada con dicho parámetro?

\begin{itemize}
\item
  El parámetro \textbf{b} va ligado directamente a la temperatura del
  material fisionable, mientras mas pequeño es el valor de b la
  temperatura llega a ser mas alta, por lo que puede ser un factor de
  corrección o algo similar.
\item
  La propiedad física relacionada con el mismo puede ser como una
  barrera limintante para la generación de calor, quizá podria ir ligada
  al tipo de isotopo o a la pureza
\end{itemize}

    \hypertarget{b-analizar-el-efecto-de-cada-paruxe1metro}{%
\subsection{\texorpdfstring{\textbf{b)} Analizar el efecto de cada
parámetro}{b) Analizar el efecto de cada parámetro}}\label{b-analizar-el-efecto-de-cada-paruxe1metro}}

\begin{itemize}
\tightlist
\item
  \(Sn_0\): La velocidad de generación volumétrica de calor, propiedad
  característica de cada material fisionable, está relacionada
  directmente con la cantidad de calor que este genera cuando se
  fisiona, llega a tener valores demasiado altos, midiendose en MW sin
  embargo es un dato difícil e encontrár en la biblíografía.
\end{itemize}

En condiciones de radios similares pero solo ilustrativas se puede ver
que gracias a este parámetro el potencial de generación de temperatura
es muy diferente, si tomamos en cuenta el Plutonio con el Torio este
ultimo tiene un valor de \emph{Sn0} mucho mayor llegando a alcanzar
temperaturas por arriba delos 8000k en comparacion con los 4000k que
alcanzaríamos con el plutonio

    \begin{tcolorbox}[breakable, size=fbox, boxrule=1pt, pad at break*=1mm,colback=cellbackground, colframe=cellborder]
\prompt{In}{incolor}{334}{\hspace{4pt}}
\begin{Verbatim}[commandchars=\\\{\}]
\PY{n}{analisisSistema}\PY{p}{(}\PY{n}{T0}\PY{p}{,}\PY{o}{.}\PY{l+m+mi}{1}\PY{p}{,}\PY{l+m+mi}{1}\PY{p}{,}\PY{n}{Plutonio}\PY{p}{,}\PY{n}{Aluminio}\PY{p}{)}
\PY{n}{analisisSistema}\PY{p}{(}\PY{n}{T0}\PY{p}{,}\PY{o}{.}\PY{l+m+mi}{1}\PY{p}{,}\PY{l+m+mi}{1}\PY{p}{,}\PY{n}{Torio}\PY{p}{,}\PY{n}{Aluminio}\PY{p}{)}
\end{Verbatim}
\end{tcolorbox}

    \begin{center}
    \adjustimage{max size={0.9\linewidth}{0.9\paperheight}}{output_13_0.png}
    \end{center}
    { \hspace*{\fill} \\}
    
    \begin{center}
    \adjustimage{max size={0.9\linewidth}{0.9\paperheight}}{output_13_1.png}
    \end{center}
    { \hspace*{\fill} \\}
    
    \begin{itemize}
\tightlist
\item
  \(R^F\): El radio del elemento fisionable responsable indirecto de la
  cantidad de calor que se producirá, dependiento del volumen total será
  la cantidad de calor que se produzca, siendo esto muy importante para
  no exceder los límites de volumen permitidos y que no se genere una
  catástrofe.
\end{itemize}

Tomando la función para dos mismos elementos pero con diferente valor en
los radios podemos encontrar correlaciones bastante interesantes. A un
mayor Rf tenemos una generación de temperatura mucho mayor, sin embargo
los valores de nuestro Rc deben de cambiar con este para no alcanzar una
zona de temperatura de riesgo, creo en lo personal que uno de los
límites para las condiciones de operación de estos sisemas es cuanto
material fisionable puedas conseguir o que puedas manejar dentro de
parámetros seguros. En este ejemplo la variación del radio fue de apenas
0.003 y tuvimos un cambio de mas de 1000K

    \begin{tcolorbox}[breakable, size=fbox, boxrule=1pt, pad at break*=1mm,colback=cellbackground, colframe=cellborder]
\prompt{In}{incolor}{342}{\hspace{4pt}}
\begin{Verbatim}[commandchars=\\\{\}]
\PY{n}{analisisSistema}\PY{p}{(}\PY{n}{T0}\PY{p}{,}\PY{o}{.}\PY{l+m+mi}{017}\PY{p}{,}\PY{o}{.}\PY{l+m+mi}{1}\PY{p}{,}\PY{n}{Uranio}\PY{p}{,}\PY{n}{Aluminio}\PY{p}{)}
\PY{n}{analisisSistema}\PY{p}{(}\PY{n}{T0}\PY{p}{,}\PY{o}{.}\PY{l+m+mi}{02}\PY{p}{,}\PY{o}{.}\PY{l+m+mi}{1}\PY{p}{,}\PY{n}{Uranio}\PY{p}{,}\PY{n}{Aluminio}\PY{p}{)}
\end{Verbatim}
\end{tcolorbox}

    \begin{center}
    \adjustimage{max size={0.9\linewidth}{0.9\paperheight}}{output_15_0.png}
    \end{center}
    { \hspace*{\fill} \\}
    
    \begin{center}
    \adjustimage{max size={0.9\linewidth}{0.9\paperheight}}{output_15_1.png}
    \end{center}
    { \hspace*{\fill} \\}
    
    \begin{itemize}
\tightlist
\item
  \(R^C\): El Radio de la cubierta de aluminio tiene la función de ser
  una zona en la que se transfiere el calor al medio refrigerante y que
  este ultimo no esté en contacto directo con el medio, según el modelo
  obtenido no es posible que el radio del material fisionable sea mayor
  al de la cubierta.
\end{itemize}

Este parámetro influye principalmente en la disipación de calor del
material fisionable al medio donde este se encuentre sumergido, cuenta
con algunas variaciones en la temperatura del material fisionable pero
su principal funcion es proteger el sistema del material fisionable y
disipar el calor de una manera mas eficiente. Un detalle que se observó
es que en el modelo cuando el radio del material de cubierta es menor al
fisionable empiezan a ocurrir discrepancias en nuestras gráficas.

    \begin{tcolorbox}[breakable, size=fbox, boxrule=1pt, pad at break*=1mm,colback=cellbackground, colframe=cellborder]
\prompt{In}{incolor}{356}{\hspace{4pt}}
\begin{Verbatim}[commandchars=\\\{\}]
\PY{n}{analisisSistema}\PY{p}{(}\PY{n}{T0}\PY{p}{,}\PY{o}{.}\PY{l+m+mi}{013}\PY{p}{,}\PY{o}{.}\PY{l+m+mi}{02}\PY{p}{,}\PY{n}{Plutonio}\PY{p}{,}\PY{n}{Aluminio}\PY{p}{)}
\PY{n}{analisisSistema}\PY{p}{(}\PY{n}{T0}\PY{p}{,}\PY{o}{.}\PY{l+m+mi}{013}\PY{p}{,}\PY{o}{.}\PY{l+m+mi}{05}\PY{p}{,}\PY{n}{Plutonio}\PY{p}{,}\PY{n}{Aluminio}\PY{p}{)}
\PY{n}{analisisSistema}\PY{p}{(}\PY{n}{T0}\PY{p}{,}\PY{o}{.}\PY{l+m+mi}{013}\PY{p}{,}\PY{o}{.}\PY{l+m+mi}{0005}\PY{p}{,}\PY{n}{Plutonio}\PY{p}{,}\PY{n}{Aluminio}\PY{p}{)}
\end{Verbatim}
\end{tcolorbox}

    \begin{center}
    \adjustimage{max size={0.9\linewidth}{0.9\paperheight}}{output_17_0.png}
    \end{center}
    { \hspace*{\fill} \\}
    
    \begin{center}
    \adjustimage{max size={0.9\linewidth}{0.9\paperheight}}{output_17_1.png}
    \end{center}
    { \hspace*{\fill} \\}
    
    \begin{center}
    \adjustimage{max size={0.9\linewidth}{0.9\paperheight}}{output_17_2.png}
    \end{center}
    { \hspace*{\fill} \\}
    
    \begin{itemize}
\tightlist
\item
  \(T_0\) Para la temperatura en la cubierta se consideró la temperatura
  promedio a la que se encuentran los refrigerantes en los que se
  sumergen este tipo de materiales, aunque puede que esto no sea del
  todo correcto, ya que el elemento que genera el calor es el material
  fisionable, este calor viaja através de la cubierta y llega así a su
  capa exterior donde despues se dispersa hacia el refrigerante.
\end{itemize}

Dependiendo de la capacidad del refrigerante en nuestro sistema influye
muchísimo la temperatura del sistema donde se encuentre, en otras
palabras la temperatura de la cubierta, ya que este ayuda en la
facilidad o no con la que un sistema alcanza temperaturas altas.

    \begin{tcolorbox}[breakable, size=fbox, boxrule=1pt, pad at break*=1mm,colback=cellbackground, colframe=cellborder]
\prompt{In}{incolor}{361}{\hspace{4pt}}
\begin{Verbatim}[commandchars=\\\{\}]
\PY{n}{analisisSistema}\PY{p}{(}\PY{l+m+mf}{273.15}\PY{p}{,}\PY{o}{.}\PY{l+m+mi}{0274}\PY{p}{,}\PY{o}{.}\PY{l+m+mi}{043}\PY{p}{,}\PY{n}{Torio}\PY{p}{,}\PY{n}{Aluminio}\PY{p}{)}
\PY{n}{analisisSistema}\PY{p}{(}\PY{l+m+mi}{800}\PY{p}{,}\PY{o}{.}\PY{l+m+mi}{0274}\PY{p}{,}\PY{o}{.}\PY{l+m+mi}{043}\PY{p}{,}\PY{n}{Torio}\PY{p}{,}\PY{n}{Aluminio}\PY{p}{)}
\end{Verbatim}
\end{tcolorbox}

    \begin{center}
    \adjustimage{max size={0.9\linewidth}{0.9\paperheight}}{output_19_0.png}
    \end{center}
    { \hspace*{\fill} \\}
    
    \begin{center}
    \adjustimage{max size={0.9\linewidth}{0.9\paperheight}}{output_19_1.png}
    \end{center}
    { \hspace*{\fill} \\}
    
    \begin{itemize}
\tightlist
\item
  \(K^F\) La conductividad del material fisionable responsable
  directamente del transporte del calor en este material, mas allá del
  calor generado una buena conductividad(Valores altos) aseguran que el
  calor generado sea facilmente transportado através del radio de este
  material al la cubierta correspondiente.
\end{itemize}

Quizá uno de los parámetros mas dificiles de observar, la conductividad
térmica indica el transporte de calor y su facilidad con el medio, si
comparamos el Plutonio y el Uranio con el aluminio (Todos estos con K
diferentes) podemos notar una diferencia muy significativa en la forma
que el calor se conduce, notando que el parámetro del K en el aluminio
es mucho mayor ya que el calor e conserva con mayor facilidad desde el
inicio hasta el final del radio.

    \begin{tcolorbox}[breakable, size=fbox, boxrule=1pt, pad at break*=1mm,colback=cellbackground, colframe=cellborder]
\prompt{In}{incolor}{362}{\hspace{4pt}}
\begin{Verbatim}[commandchars=\\\{\}]
\PY{n}{analisisSistema}\PY{p}{(}\PY{n}{T0}\PY{p}{,}\PY{o}{.}\PY{l+m+mi}{017}\PY{p}{,}\PY{o}{.}\PY{l+m+mi}{1}\PY{p}{,}\PY{n}{Uranio}\PY{p}{,}\PY{n}{Aluminio}\PY{p}{)}
\PY{n}{analisisSistema}\PY{p}{(}\PY{n}{T0}\PY{p}{,}\PY{o}{.}\PY{l+m+mi}{017}\PY{p}{,}\PY{o}{.}\PY{l+m+mi}{1}\PY{p}{,}\PY{n}{Plutonio}\PY{p}{,}\PY{n}{Aluminio}\PY{p}{)}
\end{Verbatim}
\end{tcolorbox}

    \begin{center}
    \adjustimage{max size={0.9\linewidth}{0.9\paperheight}}{output_21_0.png}
    \end{center}
    { \hspace*{\fill} \\}
    
    \begin{center}
    \adjustimage{max size={0.9\linewidth}{0.9\paperheight}}{output_21_1.png}
    \end{center}
    { \hspace*{\fill} \\}
    
    \hypertarget{c-en-base-a-los-gruxe1ficos-obtenidos}{%
\subsection{\texorpdfstring{\textbf{c)} En base a los gráficos
obtenidos}{c) En base a los gráficos obtenidos}}\label{c-en-base-a-los-gruxe1ficos-obtenidos}}

Decir bajo que condiciones sería seguro tener el material fisionable
operando sin que este ni la cubierta de aluminio lleguen a fundirse , es
decir, que siempre se mantengan operando es estado físico solido.\\

Retomando las primeras tres graficas realizadas, hay algunos aspectos a
tomar en cuenta para una correcta operación de este tipo de sistemas:

\begin{itemize}
\item
  La temperatura del lado del material fisionable nunca puede sobrepasar
  el punto de fusión del material fisionable, puede sobrepasar el punto
  de fusión del material de cubierta siempre y cuando esto sea cercano
  al centro de las esferas y no cuando se aproxima al inicio del radio
  de cubierta, un caso bien interesante es el del Plutonio, donde su
  punto de fusión es menor al del aluminio, entonces despues de la
  interfáz Rf-Rc no se alcanzan temperaturas de riesgo para la
  integridad del sistema.
\item
  La temperatura de la cubierta no puede sobrepasar el punto de fusión
  del material de la cubierta. Usualmente este material tiene a alcanzar
  el equilibrio con la temperatura del ambiente, por lo que el punto
  crítico ocurre en la interfáz Rf-Rc. Se busca mantener valores or
  debajo de las lineas amarillas y por encima de las lineas verdes,
  asegurando un intervalo de funcionamiento seguro.
\item
  La temperatura de cualquiera de los sistemas no puede ser menor a la
  T0, si esto ocurre indicaría un error, para nuestro modelo esto puede
  ocurrir ya que consideramos que el flujo de calor es unidireccional en
  r positivo pero en la realidad esto no sucede.
\item
  Otro punto interesante a tomar en cuenta sería la producción de
  radiación del equipo y como se relaciona con el volumen del sistema
  pero no es tema de estudio de este problema.
\end{itemize}

\begin{longtable}[]{@{}ll@{}}
\toprule
Elemento & Tmax(K)\tabularnewline
\midrule
\endhead
Plutonio & 912.55\tabularnewline
Uranio & 1405.15\tabularnewline
Torio & 2023.15\tabularnewline
Aluminio & 934.47\tabularnewline
\bottomrule
\end{longtable}

    \begin{tcolorbox}[breakable, size=fbox, boxrule=1pt, pad at break*=1mm,colback=cellbackground, colframe=cellborder]
\prompt{In}{incolor}{363}{\hspace{4pt}}
\begin{Verbatim}[commandchars=\\\{\}]
\PY{n}{analisisSistema}\PY{p}{(}\PY{n}{T0}\PY{p}{,}\PY{o}{.}\PY{l+m+mi}{013}\PY{p}{,}\PY{o}{.}\PY{l+m+mi}{08}\PY{p}{,}\PY{n}{Plutonio}\PY{p}{,}\PY{n}{Aluminio}\PY{p}{)}
\PY{n}{analisisSistema}\PY{p}{(}\PY{n}{T0}\PY{p}{,}\PY{o}{.}\PY{l+m+mi}{0274}\PY{p}{,}\PY{o}{.}\PY{l+m+mi}{043}\PY{p}{,}\PY{n}{Torio}\PY{p}{,}\PY{n}{Aluminio}\PY{p}{)}
\PY{n}{analisisSistema}\PY{p}{(}\PY{n}{T0}\PY{p}{,}\PY{o}{.}\PY{l+m+mi}{017}\PY{p}{,}\PY{o}{.}\PY{l+m+mi}{1}\PY{p}{,}\PY{n}{Uranio}\PY{p}{,}\PY{n}{Aluminio}\PY{p}{)}
\end{Verbatim}
\end{tcolorbox}

    \begin{center}
    \adjustimage{max size={0.9\linewidth}{0.9\paperheight}}{output_23_0.png}
    \end{center}
    { \hspace*{\fill} \\}
    
    \begin{center}
    \adjustimage{max size={0.9\linewidth}{0.9\paperheight}}{output_23_1.png}
    \end{center}
    { \hspace*{\fill} \\}
    
    \begin{center}
    \adjustimage{max size={0.9\linewidth}{0.9\paperheight}}{output_23_2.png}
    \end{center}
    { \hspace*{\fill} \\}
    
    \hypertarget{conclusiones}{%
\subsection{Conclusiones}\label{conclusiones}}

Realizar este tipo de ejercicios a nivel industrial permite ahorarse
muchisimo dinero en material así como en terminos de seguridad. Hubo
varios detalles que me llamaron la atención, como la variación de
parametros entre los radios ya que a mi parecer son los unicos valores
que podemos cambiar realmente, las demás son propiedades ya fijas de los
materiales, debe de existir una relación específica entre radios que
permita el meyor aprovechamiento de temperatura pero de una forma
segura, sin embargo a groso modo pude deducir que según el modelo
siempre se debe cumplir que Rc\textgreater{}=Rf si no el modelo empieza
a realizar cosas extrañas. Existen otros limites de operación como la
temperatura ambiente, temperatura de cubierta o T0, si nuestros valores
fueran menores a esta entonces tendriamos un error considerable. 
Otro aspecto importante es checar que la temperatura en las fronteras si
sea la misma aunque su comportamiento gracias a la K de cada material
sea direfente, esto solo indica la capacidad de transferir calor de cada
material, mucho mayor en la cubierta que en el fisionable por cierto.



    % Add a bibliography block to the postdoc
    
    
    
    \end{document}
